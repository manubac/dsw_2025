\documentclass[a4paper,12pt]{article}
\usepackage[utf8]{inputenc}
\usepackage[spanish]{babel}
\usepackage{amsmath, amssymb}
\usepackage{graphicx}
\usepackage{hyperref}
\usepackage{geometry}
\geometry{margin=2.5cm}

\title{Propuesta TP DSW: Plataforma de Compra y Distribución de Cartas Pokémon}
\author{
  Triches Alan Facundo (49598) \\
  Manuel Bacolla (50214) \\
  Volentiera Nicolás (51824) \\
  Bruno Leo Santi (51950)
}
\date{\today}

\begin{document}

\maketitle

\section*{Resumen}
Este proyecto consiste en una plataforma web de compraventa de cartas coleccionables donde los usuarios pueden explorar, reservar y adquirir cartas publicadas por otros usuarios.

El sistema introduce una lógica de compra agrupada por ciudad, optimizando la logística mediante la participación de intermediarios locales que gestionan la entrega a los compradores finales.

\section{Descripción General}
Es un marketplace especializado en la compra y distribución de cartas de Pokémon, donde tiendas oficiales agrupan pedidos de varios vendedores para enviarlos a otra tienda oficial.

La plataforma optimiza la distribución de cartas, permitiendo a los minoristas obtener mejores precios y reducir costos de envío, incentivando una dinámica colaborativa entre usuarios y vendedores.

\section{Dinámica del Sistema}

\begin{enumerate}
  \item \textbf{Publicación:} Los usuarios pueden publicar cartas indicando precio, descripción y stock.
  \item \textbf{Reserva:} Otros usuarios pueden reservar cartas sin generar envíos inmediatos.
  \item \textbf{Consolidación:} Al alcanzarse un mínimo de reservas en una ciudad, se genera un pedido agrupado.
  \item \textbf{Envío:} El pedido se envía a un intermediario designado, quien lo distribuye localmente.
  \item \textbf{Valoraciones:} Los usuarios pueden calificar a los intermediarios tras recibir sus cartas.
\end{enumerate}

\section{Objetivos}
\begin{itemize}
  \item Crear una plataforma web intuitiva y funcional.
  \item Optimizar la logística de envíos mediante agrupamiento por ciudad.
  \item Incentivar compras colaborativas para reducir costos.
  \item Facilitar la interacción entre coleccionistas y vendedores.
\end{itemize}

\section{Tecnologías}
A definir según el stack elegido por el grupo, posibles tecnologías:
\begin{itemize}
  \item Frontend: HTML, CSS, JavaScript
  \item Backend: Node.js / Python / PHP
  \item Base de datos: PostgreSQL / MySQL / MongoDB
\end{itemize}

\section{Organización del Grupo}
\begin{itemize}
  \item Triches Alan Facundo – [rol asignado]
  \item Manuel Bacolla – [rol asignado]
  \item Volentiera Nicolás – [rol asignado]
  \item Bruno Leo Santi – [rol asignado]
\end{itemize}

\section{Conclusión}
Esta plataforma busca mejorar la experiencia de compra y venta de cartas coleccionables, aprovechando un modelo colaborativo de logística y fomentando una comunidad activa de usuarios.

\end{document}